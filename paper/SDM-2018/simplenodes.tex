\section{The \variant{\childrentransitions} problem}
\label{sec:simplenodes}

In this section, we provide the formal problem definition of the 
{\variant{\childrentransitions}}
and describe a greedy polynomial-time algorithm for solving it.

\begin{problem}[{\variant{\nodeitems}}]
\label{problem:simplenodes-variant}
Given $G=(V,E)$, transition matrix $\transition$, 
initial distribution of items to nodes
$\initial$ and integer $k$, find
$S\subseteq V$ such that $|S|=k$ such that 
$\objective_{_\shortchildrentransitions}\left(S\right)$ is minimized.
\label{problem:simple-nodes-variant}
\end{problem}

As we can see from the definition of 
$\objective_\shortchildrentransitions\left(S\right)$ 
in \eqref{eq:shortchildrentransitions},
the contribution of each node $u\in {V-S}$
to the objective function is equal to
\begin{equation}
\label{eq:parent-contribution}
\objective_{_\shortchildrentransitions}(\{u\}) = 
	\initial(u)\sum_{v\in V}\transition(u,v)\left(1-\transition(u,v)\right)
\end{equation}
and independent of any other node in $V-S$.
It directly follows, then, that Problem~\ref{problem:simplenodes-variant}
is solved by a simple greedy algorithm
that places in $S$ the {\bf top k} nodes $u$ in terms
of contribution $\objective_{_\shortchildrentransitions}(u)$.
