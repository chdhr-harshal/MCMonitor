\section{Conclusions}

In this paper, we introduced the problem of \mcproblem: given
a distribution of items over a \markovchain, we aim to perform
a limited number of monitoring operations, so as to adequately
predict the position of items on the chain after one transition step.
We studied variants of this problem
and provided efficient algorithms to solve them. Our experimental evaluation
demonstrated the superiority of the proposed algorithms compared to 
baselines and the practical utility of the results in real settings.

%We see this paper as a first step of a line of works to come.
%Notably, we addressed the problem in a setting that assumed we had full 
%knowledge of item distribution before transition. 
A natural extension of this work is to to select monitoring operations under incomplete information
for initial item distribution
-- which would allow the operations to be deployed in perpetuity.
Another future-work direction is to consider different types of 
monitoring operations such as those
that combine knowledge of item placement on nodes
and edges. Finally, we can also consider more complex traffic models
(e.g., involving queuing and different transition delays between 
nodes~\cite{gallager2012discrete}).
In all cases, the algorithms we developed in this paper will serve as 
the basis for future work.




\todo[MM]{An interesting meta-question is to see which algorithm
leads to better objective function: greedy for \parentstransitions\
or for \childrentransitions?}