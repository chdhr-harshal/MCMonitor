\section{Problem Definition}
\label{section:problems}

The general problem of \mcproblem\ is to select the
appropriate monitoring operations to reduce
the expected uncertainty after they are performed.
Stated formally:

\begin{problem}[\mcproblem]
Given a transition matrix $P$ and an initial distribution of items $\initial$,
select a set of up to $k$ monitoring operations 
to minimize the expected uncertainty $\objective$. 
% $E[\uncertainty]$
\end{problem}

We study variants of the problem --
each defined for a specific type of monitoring operation.
 For simplicity, we refer to these problems
 with the same name as that of the operation type:
\variant{\parentstransitions},
\variant{\nodeitems},
\variant{\childrentransitions}, 
and \variant{\edgetransitions}.

Furthermore, as we saw in Section~\ref{sec:setting},
variants
\variant{\parentstransitions}\ and
\variant{\nodeitems} are equivalent: for the same set of nodes,
operations of the first type
reduce expected uncertainty as much as the second.
Therefore, in what follows,
we treat only the variant of \variant{\nodeitems}, as  our claims apply directly to
\variant{\parentstransitions}\ as well.

\todo[MM]{One issue we have not addressed is how we go from 
$A_{_\shortnodeitems}(S)$ to $A_{_\shortparentstransitions}(S)$.
Does it affect our claims?}
