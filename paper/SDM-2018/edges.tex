\section{The \variant{\edgetransitions} problem}
\label{sec:edges}

Whereas \variant{\nodeitems} (Problem~\ref{problem:nodes-variant}) seeks $k$ 
nodes to optimize expected uncertainty, \variant{\edgetransitions} seeks
$k$ edges.
\begin{problem}[{\edgeproblem}]
Given $G=(V,E)$, transition matrix $\transition$, 
initial distribution of items to nodes
$\initial$ and integer $k$, find
$S\subseteq V$ such that $|S|=k$ such that 
$\objective_\shortedgetransitions\left(S\right)$
(Equation~\eqref{eq:edgetransitions}) is minimized.
\label{problem:edges-variant}
\end{problem}
We provide two polynomial-time algorithms to solve
the problem, namely \edgeDP\ and \edgegreedy.
For the former, we can also prove that it is optimal, and thus Problem~\ref{problem:edges-variant}
is solvable in polynomial time.

\spara{The {\edgeDP} algorithm:}
{\edgeDP} is a dynamic-programming algorithm that 
selects edges in two steps: first, it sorts the {\it outgoing} 
edges of
each node in decreasing order of transition probability, thus
creating $|V| = n$ corresponding lists; secondly, it combines top edges from
each list to select a total of $k$ edges.
%The algorithm is polynomial and optimal.

In more detail, let $SOL_i(k)$ be the cost
of an optimal solution for the special case of
a budget of $k$ edges allocated among outgoing edges of nodes
$V_{i:} = \{i, i+1, \ldots, n\}$.
According to this notational convention,
the cost of an optimal solution $D_{opt}$ for the problem is given by 
$SOL_1(k)$.
%  -- 
% and we define $SOL_i(k) = \infty$ for $k < 0$ and $SOL_i(k) = 0$ 
% for $i > \|V\| = n$.
Moreover, considering Equation~\eqref{eq:edgetransitions},
let $\objective_i$ be the function that corresponds to
the (outer) summation term for node $i$
\begin{equation}
	\objective_i(D) = \initial^{\prime\prime}(i)\sum_{v\in V\setminus S}\transition^{\prime\prime}(i,v)\left(1-\transition^{\prime\prime}(i,v)\right)
\end{equation}
(under the auxilliary definitions of Equations \eqref{eq:etinit}
and \eqref{eq:ettransit})
and $ISOL_i(m)$ its optimal value when $D$ contains no more than
$m\leq k$ outgoing edges from node $i$.
Let also $D_i^m$ be a subset of $k$ outgoing edges of $i$
with the highest transition
probabilities.
It can be shown that the optimal value
for $\objective_i(D)$ is achieved for 
the edges $D_i^k$ with {\it highest} transition 
probability\footnote{For a proof,
see Supplementary Material, Lemma~\ref{lemma:singlenodeoptimality}.}.
Having the outgoing edges of $i$ sorted by transition probability,
we can compute $ISOL_i(m)$ for all $m = 0\ldots k$.

The dynamic programming equation is: 
\begin{equation}
	SOL_i(k) = \argmin_{0\leq m\leq k}\{ISOL_i(m) + SOL_{i+1}(k - m)\}
	\label{eq:dptop}
\end{equation}
\edgeDP\ essentially computes and keeps in memory 
$\|V\| \times (k+1)$ values according to Eq.\eqref{eq:dptop}.

Given the above, we have the following result:

\begin{lemma}
The {\edgeDP} algorithm is optimal for the {\edgeproblem} problem.
\end{lemma}

% \begin{algorithm}
% \LinesNumbered
%  \KwIn{k}
%  \KwOut{SOL: Dynamic programming array}
%  initialize empty array $SOL_{\|V\| \times (k+1)}$;\\
%  \For{i = $\|V\|$..1}{
%   \For{$k'$ = 0:k}{
%     SOL[$i$, $k'$] := $\argmin_{0\leq k_i\leq k'}\{ISOL_i(k_i) + SOL[i+1, k' - k_i]\}\}$
%    }
%  }
%  \Return SOL;
%  \caption{Dynamic programming algorithm for the \edgetransitions\ variant.}
%  \label{alg:edgeDP}
% \end{algorithm}

For a proof sketch of this lemma see Supplementary Material, 
Theorem~\ref{theorem:edgeDP}.


\emph{Running time:} \edgeDP\ 
computes $k\times |V|$ values. For each value to be
computed, up to $O(k)$ numerical operations are performed. Therefore,
\edgeDP\ runs in $O(k^2|V|)$ operations. Backtracking to retrieve the
optimal solution requires at most equal number of steps, so it does not
increase the asymptotic running time.





\spara{The {\edgegreedy} algorithm:}
\edgegreedy\ is a natural greedy algorithm that selects
$k$ edges in an equal number of steps, in each step selecting
one more edge to minimize $\objective_{\shortedgetransitions}$.
% \edgegreedy\ is described in Algorithm~\ref{alg:edgegreedy}.


% \begin{algorithm}
% \LinesNumbered
%  \KwIn{k}
%  \KwOut{$ResultEdges$: Set of selected edges}
%  $ResultEdges$ = $\{\}$ \\
%  \For{$i = 1 \cdots k$}{
%  	Select $e\in E$ := 
%  		$\argmin \objective_{\shortedgetransitions}(ResultEdges \cup \{e\})$ \\
%  	$ResultEdges$ := $ResultEdges$ $\cup$ \{e\}; \\
%  	E := E $\setminus$ \{e\}; \\
%  }
%  \Return $ResultEdges$;
%  \caption{Greedy algorithm for the \edgetransitions\ variant.}
%  \label{alg:edgegreedy}
% \end{algorithm}


%Unfortunately, we have not been able to prove or dispove optimality
%for \edgegreedy. 

In all our experiments the performance of {\edgegreedy} is the same as the performance of
the optimal {\edgeDP} algorithm. However, we do not have a proof that the former is also optimal. We leave this as a problem for future work.

\emph{Running time:} Following Equation \eqref{eq:edgetransitions},
to select $k$ edges,
\edgegreedy\ %as defined in Algorithm~\ref{alg:edgegreedy}
invokes up to $k\times O(|E|)$ evaluations of
$\objective_{\shortedgetransitions}$.
% ; and each evaluation of
% $\objective_{\shortedgetransitions}$ involves $O(|E|)$ numerical operations.
As we discussed for \nodegreedy, if the evaluation of the 
objective function is naively implemented with a double summation,
the running time of \edgegreedy\ is 
$O(k|E||V|^2)$ numerical operations.
If the objective function is implemented as a summation over edges,
the running time improves to $O(k|E|^2)$.
Furthermore, following the observations similar to those we saw for {\nodegreedy}, the
running time of \edgegreedy\ becomes $O(|E| + k|E|) = O(k|E|)$.

We notice that \edgeDP\ has better
performance than \edgegreedy\ for dense graphs ($|E|\approxeq |V|^2$) 
and small $k$. Moreover, as with \nodegreedy, \edgegreedy\ is 
amenable to parallelization - the new value of the objective function
can be computed in independently for each edge that's considered for
selection.



% We have the following theorem for \edgegreedy.
% \begin{theorem}
% \edgegreedy\ is optimal for the \edgetransitions\ variant of 
% the \mcproblem\ problem.
% \label{theorem:edgegreedy}
% \end{theorem}
% The proof for Theorem~\ref{theorem:edgegreedy} mirrors the proof for 
% Theorem~\ref{theorem:nodegreedy}, adjusted to the selection of edges instead
% of nodes - and is omitted for brevity.


